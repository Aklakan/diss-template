
\section*{Vorwort zur Reihe Publikationen in der Informatik}
\addcontentsline{toc}{chapter}{Preface by Thomas Riechert}

Mit dieser Dissertation von Natanael Arndt startet das Institut für Informatik an der Hochschule für Technik, Wirtschaft und Kultur Leipzig die Buchreihe „Publikationen in der Informatik“.

Ziel der Buchreihe ist zu aktuellen Forschungsthemen am Institut für Informatik mit seinen Kooperationspartnern, in Form von Forschungsberichten, Herausgeberbänden, Dissertationen, sowie herausragenden studentischen Abschlussarbeiten zu publizieren. Die Buchreihe erscheint bei dem sich aktuell in Gründung befindenden Open-Access-Hochschulverlag Leipzig und richtet sich an die internationale Forschungsgemeinschaft, Anwender aus der Industrie und Wirtschaft, sowie Lehrende und Studierende.

Wir freuen uns mit der Dissertationsschrift „Distributed Collaboration on Versioned Decentralized RDF Knowledge Bases“ von Natanael Arndt die Buchreihe zu starten. Herr Arndt promovierte in einem kooperativen Verfahren an der Fakultät Mathematik und Informatik der Universität Leipzig in Kooperation mit der Hochschule für Technik, Wirtschaft und Kultur Leipzig. Seine wissenschaftlichen Arbeiten entstanden innerhalb verschieden geförderter Forschungsprojekte, an der Universität Leipzig, der Sächsischen Landes- und Universitätsbibliothek (SLUB), dem Institut für Abgewandte Informatik (InfAI) und der Hochschule für Technik, Wirtschaft und Kultur Leipzig.

Innerhalb der Forschungsgruppe Agile Knowledge Engineering and Semantic Web (AKSW) liegt sein Forschungsschwerpunkt im Bereich des Semantic Web und des Wissensmanagement. Die Ergebnisse seiner Forschungen zum Thema verteiltes und kollaboratives Erstellen, Bearbeiten und Publizieren von RDF-Wissensdatenbanken finden Anwendung u.a. bei Kollaboration zwischen Partnern mit gemeinsamen Forschungsdaten, der Publikation von Forschungsdatenbanken und Forschungsinformationen, sowie bei der Integration organisationsübergreifender Geschäftsprozesse.
Natanael Arndt transferiert seine aktuellen Forschungsergebnisse darüber hinaus in die Lehre an der Hochschule. So ist er als Dozent in den Modulen Semantic Web und Media Life-Cycle Management in den Masterstudiengängen Informatik und Medieninformatik tätig.
\vspace{1cm}

\noindent
Prof. Dr. Thomas Riechert\hfill Leipzig, Juli 2020\\
Institut für Informatik\\
Fakultät Informatik und Medien\\
Hochschule für Technik, Wirtschaft und Kultur Leipzig (HTWK)

\clearpage

\section*{Preface to the Series Publications in Computer Science}

With this dissertation by Natanael Arndt, the Institute for Computer Science at the Leipzig University of Applied Sciences (Hochschule für Technik, Wirtschaft und Kultur Leipzig) starts the book series “Publications in Computer Science”.


The goal of the book series is to publish on current research topics at the Institute of Computer Science and its cooperation partners in the form of research reports, edited volumes, dissertations, and outstanding student theses. The book series is published by the Open Access University Press Leipzig (Open-Access-Hochschulverlag Leipzig), which is currently in the process of being founded. The book series is aimed at the international research community, users from industry and commerce, as well as teachers and students.

We are pleased to start the book series with the dissertation “Distributed Collaboration on Versioned Decentralized RDF Knowledge Bases” by Natanael Arndt. Natanael Arndt received his doctorate at the Faculty of Mathematics and Computer Science at the Leipzig University in cooperation with the Leipzig University of Applied Sciences. His scientific work was developed within variously funded research projects, at the Leipzig University, the Saxon State and University Library (SLUB), the Institute for Applied Informatics (InfAI) and the Leipzig University of Applied Sciences.

Within the research group Agile Knowledge Engineering and Semantic Web (AKSW) his research focus is on the Semantic Web and knowledge engineering. The results of his research on distributed and collaborative creation, editing, and publishing of RDF knowledge bases are applied to collaborations between partners with shared research data, the publication of research databases and research information, and the integration of cross-organizational business processes.
Natanael Arndt also transfers his current research results to teaching at the university. For example, he is a lecturer in the Semantic Web and Media Life-Cycle Management modules in the master's programs in Computer Science and Media Informatics.

\vspace{1cm}

\noindent
Prof. Dr. Thomas Riechert\hfill Leipzig, July 2020\\
Institute for Computer Science\\
Faculty of Computer Science and Media\\
Leipzig University of Applied Sciences (HTWK Leipzig)

\clearpage
\pagestyle{empty}
\null
\clearpage
\pagestyle{headings}

\section*{Preface to the Dissertation}
\addcontentsline{toc}{chapter}{Preface by Cesare Pautasso}

This book is about how to transfer selected methods, tools and techniques from software engineering. You will learn how to successfully apply them to the domain of knowledge engineering by crowdsourcing the maintenance of globally distributed knowledge bases. The book opens with an easy to read overview over the state of the art of the Semantic Web, including the resource description framework (RDF), its SPARQL query language and protocol, Linked Data, as well as a gentle introduction to how the git distributed version control system works.

The cross fertilization of knowledge engineering with software engineering is motivated by the need to strengthen the support provided to knowledge engineers as they need to manage the whole lifecycle of linked datasets whose content gets created, revised and published by more and more authors all over the World Wide Web. The collaborative knowledge engineering lifecycle is complex, with requirements such as: redundant distribution, asynchronous editing and review, replica synchronization, lineage tracking, conflict detection and resolution. You will discover how solutions that work well within software development teams can also bring an advantage for collaborative knowledge engineering.

In an industry where re-inventing the wheel is still happening too often, Dr. Arndt proposal to re-establish a foundation for the semantic Web technology stack based on distributed version control is both refreshing and very welcome. Likewise, containers should definitely be reused to simplify and reduce the cost of publication with good availability guarantees of linked data knowledge bases. And continuous integration build pipelines are indeed highly suitable to transform machine-to-machine representations into more human-friendly ones to facilitate both the consumption and the curation of the Web of Data.  In particular, the Jekyll RDF tool is spot on, both in terms of the proposed continuous integration methodology, but also of how the lightweight template-based approach can help to generate customized, user-friendly Websites directly from knowledge bases. You will find it particularly interesting to read about the proposal of how to ride the ongoing “re-decentralization of the Web” trend and apply federated social media solutions to support the discussion and collaboration over knowledge bases at Web scale.

As you read the manuscript, you will reflect together with Dr. Natanael Arndt on what is it that turns “linked data” into “knowledge bases”.
%
Does the answer lie in the triple- or quad-based representation of its semantics?
%
Or is it because links allow to spontaneously and freely connect multiple independent datasets scattered all over the Web?
%
Or, as suggested by Dr. Arndt, it is because of our scholastic culture of collaborative exchange of information, where different, truth-seeking minds should be allowed to dissent while having the means to discuss and resolve conflicts (disputationes de quodlibet) by rationally building agreement and consensus around the knowledge they collect and share.
%
Making one more step towards answering such timeless question, this book proposes a novel and useful collection of information processing tools that you should consider adopting
on your way to realize the vision of what could soon evolve into a Wikipedia for the semantic Web.

\vspace{1cm}

\noindent
Prof. Dr. Cesare Pautasso\hfill Lugano, July 2020\\
Software Institute\\
Faculty of Informatics\\
University of Lugano (USI)
