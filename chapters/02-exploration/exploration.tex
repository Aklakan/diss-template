% Exploration
\chapter{Knowledge Graph Exploration}
As Faceted Search is an established means for exploring catalogs, such as products and images,
by means of restricting a result set to those items that match a set of constraints.
The question is to what extent can the Faceted Search paradigm be adapted to operate on KGs via SPARQL.
Also, operations known from Business Intelligence systems and Data Cubes (e.g. drill down) are very relevant topics
that need to be investigated.
\todo{Faceted Search basics, SPARQL-based faceted search concepts, Facete implementation, LinkedGeoData use case.}

\section{Introduction - Search Paradigms}

\section{Related Work}
\todo{Consolidate work}

\section{SPARQL-based Faceted Search}
{
\let\section\subsection
\let\subsection\subsubsection
\subimport{./}{faceted-search}
}

\section{Exploring SPARQL Endpoints with Facete}
{
\let\section\subsection
\let\subsection\subsubsection
\subimport{./}{facete}
}

\section{Case Study: LinkedGeoData}
{
\let\section\subsection
\let\subsection\subsubsection
\subimport{./}{linkedgeodata}
}

\section{Discussion}

